\documentclass[journal,12pt,twocolumn]{IEEEtran}

\usepackage{setspace}
\usepackage{gensymb}

\singlespacing


\usepackage[cmex10]{amsmath}

\usepackage{amsthm}

\usepackage{mathrsfs}
\usepackage{txfonts}
\usepackage{stfloats}
\usepackage{bm}
\usepackage{cite}
\usepackage{cases}
\usepackage{subfig}

\usepackage{longtable}
\usepackage{multirow}

\usepackage{enumitem}
\usepackage{mathtools}
\usepackage{steinmetz}
\usepackage{tikz}
\usepackage{circuitikz}
\usepackage{verbatim}
\usepackage{tfrupee}
\usepackage[breaklinks=true]{hyperref}
\usepackage{graphicx}
\usepackage{tkz-euclide}

\usetikzlibrary{calc,math}
\usepackage{listings}
    \usepackage{color}                                            %%
    \usepackage{array}                                            %%
    \usepackage{longtable}                                        %%
    \usepackage{calc}                                             %%
    \usepackage{multirow}                                         %%
    \usepackage{hhline}                                           %%
    \usepackage{ifthen}                                           %%
    \usepackage{lscape}     
\usepackage{multicol}
\usepackage{chngcntr}

\DeclareMathOperator*{\Res}{Res}

\renewcommand\thesection{\arabic{section}}
\renewcommand\thesubsection{\thesection.\arabic{subsection}}
\renewcommand\thesubsubsection{\thesubsection.\arabic{subsubsection}}

\renewcommand\thesectiondis{\arabic{section}}
\renewcommand\thesubsectiondis{\thesectiondis.\arabic{subsection}}
\renewcommand\thesubsubsectiondis{\thesubsectiondis.\arabic{subsubsection}}


\hyphenation{op-tical net-works semi-conduc-tor}
\def\inputGnumericTable{}                                 %%

\lstset{
%language=C,
frame=single, 
breaklines=true,
columns=fullflexible
}
\begin{document}


\newtheorem{theorem}{Theorem}[section]
\newtheorem{problem}{Problem}
\newtheorem{proposition}{Proposition}[section]
\newtheorem{lemma}{Lemma}[section]
\newtheorem{corollary}[theorem]{Corollary}
\newtheorem{example}{Example}[section]
\newtheorem{definition}[problem]{Definition}

\newcommand{\BEQA}{\begin{eqnarray}}
\newcommand{\EEQA}{\end{eqnarray}}
\newcommand{\define}{\stackrel{\triangle}{=}}
\bibliographystyle{IEEEtran}
\providecommand{\mbf}{\mathbf}
\providecommand{\pr}[1]{\ensuremath{\Pr\left(#1\right)}}
\providecommand{\qfunc}[1]{\ensuremath{Q\left(#1\right)}}
\providecommand{\sbrak}[1]{\ensuremath{{}\left[#1\right]}}
\providecommand{\lsbrak}[1]{\ensuremath{{}\left[#1\right.}}
\providecommand{\rsbrak}[1]{\ensuremath{{}\left.#1\right]}}
\providecommand{\brak}[1]{\ensuremath{\left(#1\right)}}
\providecommand{\lbrak}[1]{\ensuremath{\left(#1\right.}}
\providecommand{\rbrak}[1]{\ensuremath{\left.#1\right)}}
\providecommand{\cbrak}[1]{\ensuremath{\left\{#1\right\}}}
\providecommand{\lcbrak}[1]{\ensuremath{\left\{#1\right.}}
\providecommand{\rcbrak}[1]{\ensuremath{\left.#1\right\}}}
\theoremstyle{remark}
\newtheorem{rem}{Remark}
\newcommand{\sgn}{\mathop{\mathrm{sgn}}}
\providecommand{\abs}[1]{\left\vert#1\right\vert}
\providecommand{\res}[1]{\Res\displaylimits_{#1}} 
\providecommand{\norm}[1]{\left\lVert#1\right\rVert}
%\providecommand{\norm}[1]{\lVert#1\rVert}
\providecommand{\mtx}[1]{\mathbf{#1}}
\providecommand{\mean}[1]{E\left[ #1 \right]}
\providecommand{\fourier}{\overset{\mathcal{F}}{ \rightleftharpoons}}
%\providecommand{\hilbert}{\overset{\mathcal{H}}{ \rightleftharpoons}}
\providecommand{\system}{\overset{\mathcal{H}}{ \longleftrightarrow}}
	%\newcommand{\solution}[2]{\textbf{Solution:}{#1}}
\newcommand{\solution}{\noindent \textbf{Solution: }}
\newcommand{\cosec}{\,\text{cosec}\,}
\providecommand{\dec}[2]{\ensuremath{\overset{#1}{\underset{#2}{\gtrless}}}}
\newcommand{\myvec}[1]{\ensuremath{\begin{pmatrix}#1\end{pmatrix}}}
\newcommand{\mydet}[1]{\ensuremath{\begin{vmatrix}#1\end{vmatrix}}}
\numberwithin{equation}{subsection}
\makeatletter
\@addtoreset{figure}{problem}
\makeatother
\let\StandardTheFigure\thefigure
\let\vec\mathbf
\renewcommand{\thefigure}{\theproblem}
\def\putbox#1#2#3{\makebox[0in][l]{\makebox[#1][l]{}\raisebox{\baselineskip}[0in][0in]{\raisebox{#2}[0in][0in]{#3}}}}
     \def\rightbox#1{\makebox[0in][r]{#1}}
     \def\centbox#1{\makebox[0in]{#1}}
     \def\topbox#1{\raisebox{-\baselineskip}[0in][0in]{#1}}
     \def\midbox#1{\raisebox{-0.5\baselineskip}[0in][0in]{#1}}
\vspace{3cm}
\title{EE5609 Matrix Theory}
\author{Kranthi Kumar P}
\date{September 2020}
\maketitle
\newpage
\bigskip
\renewcommand{\thefigure}{\theenumi}
\renewcommand{\thetable}{\theenumi}
%Download the python code for from 
%\begin{lstlisting}
%https://github.com/kranthiakssy/AI20RESCH14002_PhD_IITH/tree/master/EE5609_Matrix_Theory/Assignment-9
%\end{lstlisting}

Download the latex-file codes from 
\begin{lstlisting}
https://github.com/kranthiakssy/AI20RESCH14002_PhD_IITH/tree/master/EE5609_Matrix_Theory/Assignment-11
\end{lstlisting}
\section*{Assignment-11}
\subsection*{Problem:}
Linear Transformations:\\
Let $\mathbb{V}$ be a vector space and $\vec{T}$ a linear transformation from $\mathbb{V}$ into $\mathbb{V}$. Prove that the following two statements about $\vec{T}$ are equivalent.
\begin{enumerate}[label = (\alph*)]
\item The intersection of the range of $\vec{T}$ and null space of $\vec{T}$ is the zero subspace of $\mathbb{V}$.
\item If $\vec{T}(\vec{T}\alpha) = 0$, then $\vec{T}\alpha = 0$.
\end{enumerate}
\subsection*{Solution:}
\begin{table}[h!]
\begin{center}
\begin{tabular}{|c|c|}
\hline
& \\
Given & $\vec{T}:\mathbb{V} \rightarrow \mathbb{V}$\\
&\\
\hline
& \\
To prove & a) $ range(\vec{T}) \cap null space(\vec{T}) = \left\lbrace 0 \right\rbrace$\\
& \\
& b) If $\vec{T}(\vec{T}\alpha) = 0$, then $\vec{T}\alpha = 0$.\\
& \\
\hline
\end{tabular}
\end{center}
\end{table}
\begin{table}[h!]
\begin{center}
\begin{tabular}{|c|c|}
\hline
& \\
Proof(a) & Let $\vec{x} \in \mathbb{V}$\\
& and \\
& $\vec{x} \in range(\vec{T}) \cap null space(\vec{T})$\\
& then,\\
& $\vec{x} \in range(\vec{T})$\\
& $\vec{x} \in null space(\vec{T})$\\
& \\
& Consider $\vec{y} \in \mathbb{V}$ whose\\
& linear transformation into $\mathbb{V}$ is $\vec{x}$.\\
& $\implies \vec{T}(\vec{y}) = \vec{x}$\\
& \\
& since $\vec{x} \in $ null space$(\vec{T})$ \\
& and the sub space is linearly independent \\
& $\vec{T}(\vec{x}) = 0$\\
& from above equations \\
& $\vec{T}(\vec{T}(y)) = 0$\\
& \\
& from the definition of linear\\
& transformation of independent vector space\\
& $\vec{T}(y) = 0$\\
& $\implies \vec{x} = 0$\\
& $\implies \left\lbrace 0 \right\rbrace \subseteq range(\vec{T}) \cap null space(\vec{T})$ \\
& $\therefore range(\vec{T}) \cap null space(\vec{T}) = \left\lbrace 0 \right\rbrace$\\
& Hence Proved.\\
& \\
\hline
& \\
Proof(b) & If $\vec{T}(\vec{T}\alpha) = 0$\\
& then, from the definition of linear\\
& transformation, $\vec{T}\alpha$ will \\
& be in the null space of linear \\
& transformation $\vec{T}$ and is linearly \\
& independent \\
& $\therefore \vec{T}\alpha = 0$\\
& \\
\hline
\end{tabular}
\end{center}
\end{table}

\begin{table}[h!]
\begin{center}
\begin{tabular}{|c|c|}
\hline
& \\
Eg: & Let $\alpha \in \mathbb{V}$ and\\
& \\
& $\alpha = \myvec{1&7&-1&-1\\-1&1&2&1\\4&-2&0&-4\\2&3&4&-2}$\\
& \\
& linear transformation of $\alpha$ into $\mathbb{V}$\\
& $\vec{T}(\alpha) = c\alpha$\\
& \\
& then row reduced echelon form of $\vec{T}$ is\\
& $rref(\vec{T}) = \myvec{1&0&0&-1\\0&1&0&0\\0&0&1&0\\0&0&0&0}$\\
& \\
& $\implies rank(\vec{T}) = 3,$\\
& $nullity(\vec{T}) = 1$\\
& \\
& $\implies range(\vec{T}) = \myvec{1&7&-1\\-1&1&2\\4&-2&0\\2&3&4},$\\
& $null space(\vec{T}) = \myvec{1\\0\\0\\1}$\\
& \\
& $\therefore range(\vec{T}) \cap null space(\vec{T}) = \left\lbrace 0 \right\rbrace$\\
& \\
& Hence proved that the intersection \\
& of the range of $\vec{T}$ and \\
& null space of $\vec{T}$ is the zero \\
& subspace of $\mathbb{V}$.\\
& \\
\hline
\end{tabular}
\end{center}
\end{table}

\end{document}