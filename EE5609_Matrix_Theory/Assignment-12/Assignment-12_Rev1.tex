\documentclass[journal,12pt]{IEEEtran}
\usepackage{longtable}
\usepackage{setspace}
\usepackage{gensymb}
\singlespacing
\usepackage[cmex10]{amsmath}
\newcommand\myemptypage{
	\null
	\thispagestyle{empty}
	\addtocounter{page}{-1}
	\newpage
}
\usepackage{amsthm}
\usepackage{mdframed}
\usepackage{mathrsfs}
\usepackage{txfonts}
\usepackage{stfloats}
\usepackage{bm}
\usepackage{cite}
\usepackage{cases}
\usepackage{subfig}

\usepackage{longtable}
\usepackage{multirow}

\usepackage{enumitem}
\usepackage{mathtools}
\usepackage{steinmetz}
\usepackage{tikz}
\usepackage{circuitikz}
\usepackage{verbatim}
\usepackage{tfrupee}
\usepackage[breaklinks=true]{hyperref}
\usepackage{graphicx}
\usepackage{tkz-euclide}

\usetikzlibrary{calc,math}
\usepackage{listings}
    \usepackage{color}                                            %%
    \usepackage{array}                                            %%
    \usepackage{longtable}                                        %%
    \usepackage{calc}                                             %%
    \usepackage{multirow}                                         %%
    \usepackage{hhline}                                           %%
    \usepackage{ifthen}                                           %%
    \usepackage{lscape}     
\usepackage{multicol}
\usepackage{chngcntr}

\DeclareMathOperator*{\Res}{Res}

\renewcommand\thesection{\arabic{section}}
\renewcommand\thesubsection{\thesection.\arabic{subsection}}
\renewcommand\thesubsubsection{\thesubsection.\arabic{subsubsection}}

\renewcommand\thesectiondis{\arabic{section}}
\renewcommand\thesubsectiondis{\thesectiondis.\arabic{subsection}}
\renewcommand\thesubsubsectiondis{\thesubsectiondis.\arabic{subsubsection}}


\hyphenation{op-tical net-works semi-conduc-tor}
\def\inputGnumericTable{}                                 %%

\lstset{
%language=C,
frame=single, 
breaklines=true,
columns=fullflexible
}
\begin{document}
\onecolumn

\newtheorem{theorem}{Theorem}[section]
\newtheorem{problem}{Problem}
\newtheorem{proposition}{Proposition}[section]
\newtheorem{lemma}{Lemma}[section]
\newtheorem{corollary}[theorem]{Corollary}
\newtheorem{example}{Example}[section]
\newtheorem{definition}[problem]{Definition}

\newcommand{\BEQA}{\begin{eqnarray}}
\newcommand{\EEQA}{\end{eqnarray}}
\newcommand{\define}{\stackrel{\triangle}{=}}
\bibliographystyle{IEEEtran}
\raggedbottom
\setlength{\parindent}{0pt}
\providecommand{\mbf}{\mathbf}
\providecommand{\pr}[1]{\ensuremath{\Pr\left(#1\right)}}
\providecommand{\qfunc}[1]{\ensuremath{Q\left(#1\right)}}
\providecommand{\sbrak}[1]{\ensuremath{{}\left[#1\right]}}
\providecommand{\lsbrak}[1]{\ensuremath{{}\left[#1\right.}}
\providecommand{\rsbrak}[1]{\ensuremath{{}\left.#1\right]}}
\providecommand{\brak}[1]{\ensuremath{\left(#1\right)}}
\providecommand{\lbrak}[1]{\ensuremath{\left(#1\right.}}
\providecommand{\rbrak}[1]{\ensuremath{\left.#1\right)}}
\providecommand{\cbrak}[1]{\ensuremath{\left\{#1\right\}}}
\providecommand{\lcbrak}[1]{\ensuremath{\left\{#1\right.}}
\providecommand{\rcbrak}[1]{\ensuremath{\left.#1\right\}}}
\theoremstyle{remark}
\newtheorem{rem}{Remark}
\newcommand{\sgn}{\mathop{\mathrm{sgn}}}
\providecommand{\abs}[1]{\left\vert#1\right\vert}
\providecommand{\res}[1]{\Res\displaylimits_{#1}} 
\providecommand{\norm}[1]{\left\lVert#1\right\rVert}
%\providecommand{\norm}[1]{\lVert#1\rVert}
\providecommand{\mtx}[1]{\mathbf{#1}}
\providecommand{\mean}[1]{E\left[ #1 \right]}
\providecommand{\fourier}{\overset{\mathcal{F}}{ \rightleftharpoons}}
%\providecommand{\hilbert}{\overset{\mathcal{H}}{ \rightleftharpoons}}
\providecommand{\system}{\overset{\mathcal{H}}{ \longleftrightarrow}}
	%\newcommand{\solution}[2]{\textbf{Solution:}{#1}}
\newcommand{\solution}{\noindent \textbf{Solution: }}
\newcommand{\cosec}{\,\text{cosec}\,}
\providecommand{\dec}[2]{\ensuremath{\overset{#1}{\underset{#2}{\gtrless}}}}
\newcommand{\myvec}[1]{\ensuremath{\begin{pmatrix}#1\end{pmatrix}}}
\newcommand{\mydet}[1]{\ensuremath{\begin{vmatrix}#1\end{vmatrix}}}
\numberwithin{equation}{subsection}
\makeatletter
\@addtoreset{figure}{problem}
\makeatother
\let\StandardTheFigure\thefigure
\let\vec\mathbf
\renewcommand{\thefigure}{\theproblem}
\def\putbox#1#2#3{\makebox[0in][l]{\makebox[#1][l]{}\raisebox{\baselineskip}[0in][0in]{\raisebox{#2}[0in][0in]{#3}}}}
     \def\rightbox#1{\makebox[0in][r]{#1}}
     \def\centbox#1{\makebox[0in]{#1}}
     \def\topbox#1{\raisebox{-\baselineskip}[0in][0in]{#1}}
     \def\midbox#1{\raisebox{-0.5\baselineskip}[0in][0in]{#1}}
\vspace{3cm}
\title{EE5609 Matrix Theory}
\author{Kranthi Kumar P}
\date{November 2020}
\maketitle
\bigskip
\renewcommand{\thefigure}{\theenumi}
\renewcommand{\thetable}{\theenumi}
%Download the python code for from 
%\begin{lstlisting}
%https://github.com/kranthiakssy/AI20RESCH14002_PhD_IITH/tree/master/EE5609_Matrix_Theory/Assignment-9
%\end{lstlisting}

Download the latex-file codes from 
\begin{lstlisting}
https://github.com/kranthiakssy/AI20RESCH14002_PhD_IITH/tree/master/EE5609_Matrix_Theory/Assignment-12
\end{lstlisting}
\section{\textbf{Problem}}
(hoffman/page84/11) : \\
Let $\mathbb{V}$ be a finite-dimensional vector space and let $\vec{T}$ be a linear operator on $\mathbb{V}$. Suppose that $rank (\vec{T}^2) = rank (\vec{T})$. Prove that the range and null space  of $\vec{T}$ are disjoint, i.e., have only the zero vector in common.
\section{\textbf{Solution}}
\renewcommand{\thetable}{1}
\begin{longtable}{|l|l|}
\hline
\multirow{3}{*}{Given} & \\
& $\mathbb{V}$ is a finite-dimensional vector space,\\
& $\vec{T}:\mathbb{V} \rightarrow \mathbb{V}$ and\\
& $ rank (\vec{T}^2) = rank (\vec{T})$\\
&\\
\hline
\multirow{3}{*}{To Prove} & \\
& range and null space of $\vec{T}$ are disjoint\\
&\\
\hline
\multirow{3}{*}{Defining $rank(\vec{T})$} & \\
& Let $\lbrace \beta_1,\ldots,\beta_k,\beta_{k+1},\ldots,\beta_n \rbrace$ is the span of $\mathbb{V}$.\\
& \\
& The linear transformation of $\mathbb{V}$ is\\
& $\lbrace \vec{T}\beta_1,\ldots,\vec{T}\beta_k,\vec{T}\beta_{k+1},\ldots,\vec{T}\beta_n \rbrace$.\\
& \\
& Suppose the $rank (\vec{T}) = k$, then \\
& the basis of $\vec{T}$ is $\lbrace \vec{T}\beta_1,\ldots,\vec{T}\beta_k \rbrace$ \\
& and are linearly independent.\\
& \\
\hline
\multirow{3}{*}{Defining $range(\vec{T}^2)$} & \\
& Now $\vec{T}^2 : \mathbb{V} \rightarrow \mathbb{V}$ be a linear transformation for any $\alpha \in \mathbb{V}$.  \\
& $\therefore \vec{T}^2(\mathbb{V}) = \vec{T}(\vec{T}(\alpha))$ and \\
& $\lbrace \vec{T}^2\beta_1,\ldots,\vec{T}^2\beta_k \rbrace$ span the range of $\vec{T}^2$\\
& \\
& since $ rank (\vec{T}^2) = rank (\vec{T})$\\
& $\implies dim \  range (\vec{T}^2) = dim \  range (\vec{T})$\\
& $\therefore \lbrace \vec{T}^2\beta_1,\ldots,\vec{T}^2\beta_k \rbrace$ must be\\
& basis for $range(\vec{T}^2)$\\
& \\
& \\
& \\
\hline
\multirow{3}{*}{Obtaining $range $} & \\
& Now let $\alpha \in range(\vec{T})$, then\\
$ and \  nullspace \  of \  \vec{T}$
& it can be written as linear combinations of \\
& vectors in $ range(\vec{T})$\\
& $\therefore \alpha = C_1\vec{T}\beta_1+C_2\vec{T}\beta_2+\ldots+C_k\vec{T}\beta_k$\\
& \\
& If $\alpha \in null space (\vec{T})$ also, then\\
& $ \vec{T}(\mathbb{V}) = 0$\\
& $\implies \vec{T}(C_1\vec{T}\beta_1+C_2\vec{T}\beta_2+\ldots+C_k\vec{T}\beta_k) = 0$\\
& $ \implies C_1\vec{T}^2\beta_1+C_2\vec{T}^2\beta_2+\ldots+C_k\vec{T}^2\beta_k = 0$\\
& \\
& since $\lbrace \vec{T}^2\beta_1,\ldots,\vec{T}^2\beta_k \rbrace$ is basis of $\vec{T}^2$\\
& $\implies C_1 = C_2 = \ldots = C_k = 0$\\
& $\implies \mathbb{V} = 0$\\
& $\therefore$ if $\alpha$  is in both $ range(\vec{T}) \  and \  null space(\vec{T})$, \\
& then $\alpha$ must be a zero vector.\\
& \\
& Hence it is proved that \\
& range and null space of $\vec{T}$ are disjoint.\\
&\\
\hline
\multirow{3}{*}{Conclusion} & \\
& The range and null space of $\vec{T}$ are disjoint.\\
&\\
\hline
\caption{Proof}
\label{table:1}
\end{longtable}
\newpage
\section{\textbf{Example}}
\renewcommand{\thetable}{2}
\begin{longtable}{|l|l|}
\hline
\multirow{3}{*}{Example} & \\
& Let $\alpha \in \mathbb{V}$ and\\
& \\
& $\alpha = \myvec{1&7&-1&-1\\-1&1&2&1\\4&-2&0&-4\\2&3&4&-2}$\\
& \\
& linear transformation of $\alpha$ into $\mathbb{V}, \vec{T}(\alpha) = c\alpha$, then row reduced echelon form of $\vec{T}$ is\\
& $rref(\vec{T}) = \myvec{1&0&0&-1\\0&1&0&0\\0&0&1&0\\0&0&0&0}$\\
& \\
& $\implies rank(\vec{T}) = 3,$\\
& $nullity(\vec{T}) = 1$\\
& \\
& $\implies range(\vec{T}) = \myvec{1&7&-1\\-1&1&2\\4&-2&0\\2&3&4},$\\
& $null space(\vec{T}) = \myvec{1\\0\\0\\1}$\\
& \\
& Now linear transformation $\vec{T}(\vec{T}(\alpha)) = cd\alpha$.\\
& \\
& Let $ c = d = 1$, then $range(\vec{T}^2) = range (\vec{T})$\\
& and $rank (\vec{T}^2) = rank (\vec{T}) = 3$,\\
& $nullity (\vec{T}^2) = nullity (\vec{T}) = 3$\\
& \\
& Hence proved that, the range and null space of $\vec{T}$ are disjoint.\\
& \\
\hline
\caption{Example}
\label{table:2}
\end{longtable}
\end{document}