\documentclass[journal,12pt]{IEEEtran}
\usepackage{longtable}
\usepackage{setspace}
\usepackage{gensymb}
\singlespacing
\usepackage[cmex10]{amsmath}
\newcommand\myemptypage{
	\null
	\thispagestyle{empty}
	\addtocounter{page}{-1}
	\newpage
}
\usepackage{amsthm}
\usepackage{mdframed}
\usepackage{mathrsfs}
\usepackage{txfonts}
\usepackage{stfloats}
\usepackage{bm}
\usepackage{cite}
\usepackage{cases}
\usepackage{subfig}

\usepackage{longtable}
\usepackage{multirow}

\usepackage{enumitem}
\usepackage{mathtools}
\usepackage{steinmetz}
\usepackage{tikz}
\usepackage{circuitikz}
\usepackage{verbatim}
\usepackage{tfrupee}
\usepackage[breaklinks=true]{hyperref}
\usepackage{graphicx}
\usepackage{tkz-euclide}

\usetikzlibrary{calc,math}
\usepackage{listings}
    \usepackage{color}                                            %%
    \usepackage{array}                                            %%
    \usepackage{longtable}                                        %%
    \usepackage{calc}                                             %%
    \usepackage{multirow}                                         %%
    \usepackage{hhline}                                           %%
    \usepackage{ifthen}                                           %%
    \usepackage{lscape}     
\usepackage{multicol}
\usepackage{chngcntr}

\DeclareMathOperator*{\Res}{Res}

\renewcommand\thesection{\arabic{section}}
\renewcommand\thesubsection{\thesection.\arabic{subsection}}
\renewcommand\thesubsubsection{\thesubsection.\arabic{subsubsection}}

\renewcommand\thesectiondis{\arabic{section}}
\renewcommand\thesubsectiondis{\thesectiondis.\arabic{subsection}}
\renewcommand\thesubsubsectiondis{\thesubsectiondis.\arabic{subsubsection}}


\hyphenation{op-tical net-works semi-conduc-tor}
\def\inputGnumericTable{}                                 %%

\lstset{
%language=C,
frame=single, 
breaklines=true,
columns=fullflexible
}
\begin{document}
\onecolumn

\newtheorem{theorem}{Theorem}[section]
\newtheorem{problem}{Problem}
\newtheorem{proposition}{Proposition}[section]
\newtheorem{lemma}{Lemma}[section]
\newtheorem{corollary}[theorem]{Corollary}
\newtheorem{example}{Example}[section]
\newtheorem{definition}[problem]{Definition}

\newcommand{\BEQA}{\begin{eqnarray}}
\newcommand{\EEQA}{\end{eqnarray}}
\newcommand{\define}{\stackrel{\triangle}{=}}
\bibliographystyle{IEEEtran}
\raggedbottom
\setlength{\parindent}{0pt}
\providecommand{\mbf}{\mathbf}
\providecommand{\pr}[1]{\ensuremath{\Pr\left(#1\right)}}
\providecommand{\qfunc}[1]{\ensuremath{Q\left(#1\right)}}
\providecommand{\sbrak}[1]{\ensuremath{{}\left[#1\right]}}
\providecommand{\lsbrak}[1]{\ensuremath{{}\left[#1\right.}}
\providecommand{\rsbrak}[1]{\ensuremath{{}\left.#1\right]}}
\providecommand{\brak}[1]{\ensuremath{\left(#1\right)}}
\providecommand{\lbrak}[1]{\ensuremath{\left(#1\right.}}
\providecommand{\rbrak}[1]{\ensuremath{\left.#1\right)}}
\providecommand{\cbrak}[1]{\ensuremath{\left\{#1\right\}}}
\providecommand{\lcbrak}[1]{\ensuremath{\left\{#1\right.}}
\providecommand{\rcbrak}[1]{\ensuremath{\left.#1\right\}}}
\theoremstyle{remark}
\newtheorem{rem}{Remark}
\newcommand{\sgn}{\mathop{\mathrm{sgn}}}
\providecommand{\abs}[1]{\left\vert#1\right\vert}
\providecommand{\res}[1]{\Res\displaylimits_{#1}} 
\providecommand{\norm}[1]{\left\lVert#1\right\rVert}
%\providecommand{\norm}[1]{\lVert#1\rVert}
\providecommand{\mtx}[1]{\mathbf{#1}}
\providecommand{\mean}[1]{E\left[ #1 \right]}
\providecommand{\fourier}{\overset{\mathcal{F}}{ \rightleftharpoons}}
%\providecommand{\hilbert}{\overset{\mathcal{H}}{ \rightleftharpoons}}
\providecommand{\system}{\overset{\mathcal{H}}{ \longleftrightarrow}}
	%\newcommand{\solution}[2]{\textbf{Solution:}{#1}}
\newcommand{\solution}{\noindent \textbf{Solution: }}
\newcommand{\cosec}{\,\text{cosec}\,}
\providecommand{\dec}[2]{\ensuremath{\overset{#1}{\underset{#2}{\gtrless}}}}
\newcommand{\myvec}[1]{\ensuremath{\begin{pmatrix}#1\end{pmatrix}}}
\newcommand{\mydet}[1]{\ensuremath{\begin{vmatrix}#1\end{vmatrix}}}
\numberwithin{equation}{subsection}
\makeatletter
\@addtoreset{figure}{problem}
\makeatother
\let\StandardTheFigure\thefigure
\let\vec\mathbf
\renewcommand{\thefigure}{\theproblem}
\def\putbox#1#2#3{\makebox[0in][l]{\makebox[#1][l]{}\raisebox{\baselineskip}[0in][0in]{\raisebox{#2}[0in][0in]{#3}}}}
     \def\rightbox#1{\makebox[0in][r]{#1}}
     \def\centbox#1{\makebox[0in]{#1}}
     \def\topbox#1{\raisebox{-\baselineskip}[0in][0in]{#1}}
     \def\midbox#1{\raisebox{-0.5\baselineskip}[0in][0in]{#1}}
\vspace{3cm}
\title{EE5609 Matrix Theory}
\author{Kranthi Kumar P}
\date{November 2020}
\maketitle
\bigskip
\renewcommand{\thefigure}{\theenumi}
\renewcommand{\thetable}{\theenumi}
%Download the python code for from 
%\begin{lstlisting}
%https://github.com/kranthiakssy/AI20RESCH14002_PhD_IITH/tree/master/EE5609_Matrix_Theory/Assignment-9
%\end{lstlisting}

Download the latex-file codes from 
\begin{lstlisting}
https://github.com/kranthiakssy/AI20RESCH14002_PhD_IITH/tree/master/EE5609_Matrix_Theory/Assignment-13
\end{lstlisting}
\section{\textbf{Problem}}
(ugcdec2014, 78) : \\
Let $\vec{A}$ be a 4 x 7 real matrix and $\vec{B}$ be a 7 x 4 real matrix such that $\vec{AB} = \vec{I}_4$, where $\vec{I}_4$ is the 4 x 4 identity matrix. Which of the following is/are always true?
\begin{enumerate} %[label = (\alph*)]
\item $rank(\vec{A}) = 4 $
\item $rank(\vec{B}) = 7 $
\item $nullity(\vec{B}) = 0 $
\item $\vec{BA} = \vec{I}_7 $, where $\vec{I}_7$ is the 7 x 7 identity matrix
\end{enumerate}
\section{\textbf{Solution}}
\renewcommand{\thetable}{1}
\begin{longtable}{|l|l|}
\hline
\multirow{3}{*}{Given} & \\
& $\vec{A}$ is 4 x 7 real matrix\\
& $\vec{B}$ is 7 x 4 real matrix\\
& $\vec{AB} =\vec{I}_4 $\\
&\\
\hline
\multirow{3}{*}{Option-1} & \\
& since $\vec{I}_4$ is a 4 x 4 identity matrix, $rank(\vec{I}_4) = 4 = rank(\vec{AB})$\\
& \\
& from the properties of matrices\\
& $rank(\vec{A}) \leq min\lbrace \# cloumns, \#rows \rbrace$\\
& $rank(\vec{A}) \leq 4$\\
& \\
& and\\
& \\
& $rank(\vec{AB}) \leq rank(\vec{A})$\\
& $ 4 \leq rank(\vec{A})$\\
& \\
& $\therefore rank(\vec{A}) = 4$\\
& Hence Option-1 is True. \\
& \\
\hline
\multirow{3}{*}{Option-2} & \\
& Similarly from the properties of matrices\\
& $rank(\vec{B}) \leq min\lbrace \# cloumns, \#rows \rbrace$\\
& $rank(\vec{B}) \leq 4$\\
& \\
& and\\
& \\
& $rank(\vec{AB}) \leq rank(\vec{B})$\\
& $ 4 \leq rank(\vec{B})$\\
& \\
& $\therefore rank(\vec{B}) = 4$\\
& Hence Option-2 is False. \\
& \\
\hline
\multirow{3}{*}{Option-3} & \\
& Since $rank(\vec{B}) = 4$, and $\vec{B}$ is a 7 x 4 matrix in \\
& finite dimensional vector space $\mathbb{V}$.\\
& the column space,$C(\vec{B})$ will form the basis.\\
& $\implies range(\vec{B}) = dim(\mathbb{V}) = 4$\\
& \\
& from rank-nullity theorem\\
& $ rank(\vec{B}) + nullity(\vec{B}) = dim(\mathbb{V})$\\
& by substituting above values\\
& $ nullity(\vec{B}) = 0$\\
& Hence Option-3 is True.\\
& \\
\hline
\multirow{3}{*}{Option-4} & \\
& Given $\vec{BA} = \vec{I}_7$\\
& $rank(\vec{I}_7) = 7 = rank(\vec{BA})$\\
& \\
& from the properties of matrices\\
& $rank(\vec{BA}) \leq rank(\vec{B})$\\
& $7 \leq rank(\vec{B})$\\
& the above conditioned can not be satisfied since we know\\
& $rank(\vec{B}) =4$.\\
& Hence Option-4 is False. \\
&\\
\hline
\multirow{3}{*}{Conclusion} & \\
& Option-1 and 3 are True\\
& Option-2 and 4 are False\\
&\\
\hline
\caption{Proof}
\label{table:1}
\end{longtable}
\end{document}