\documentclass[journal,12pt]{IEEEtran}
\usepackage{longtable}
\usepackage{setspace}
\usepackage{gensymb}
\singlespacing
\usepackage[cmex10]{amsmath}
\newcommand\myemptypage{
	\null
	\thispagestyle{empty}
	\addtocounter{page}{-1}
	\newpage
}
\usepackage{amsthm}
\usepackage{mdframed}
\usepackage{mathrsfs}
\usepackage{txfonts}
\usepackage{stfloats}
\usepackage{bm}
\usepackage{cite}
\usepackage{cases}
\usepackage{subfig}

\usepackage{longtable}
\usepackage{multirow}

\usepackage{enumitem}
\usepackage{mathtools}
\usepackage{steinmetz}
\usepackage{tikz}
\usepackage{circuitikz}
\usepackage{verbatim}
\usepackage{tfrupee}
\usepackage[breaklinks=true]{hyperref}
\usepackage{graphicx}
\usepackage{tkz-euclide}

\usetikzlibrary{calc,math}
\usepackage{listings}
    \usepackage{color}                                            %%
    \usepackage{array}                                            %%
    \usepackage{longtable}                                        %%
    \usepackage{calc}                                             %%
    \usepackage{multirow}                                         %%
    \usepackage{hhline}                                           %%
    \usepackage{ifthen}                                           %%
    \usepackage{lscape}     
\usepackage{multicol}
\usepackage{chngcntr}

\DeclareMathOperator*{\Res}{Res}

\renewcommand\thesection{\arabic{section}}
\renewcommand\thesubsection{\thesection.\arabic{subsection}}
\renewcommand\thesubsubsection{\thesubsection.\arabic{subsubsection}}

\renewcommand\thesectiondis{\arabic{section}}
\renewcommand\thesubsectiondis{\thesectiondis.\arabic{subsection}}
\renewcommand\thesubsubsectiondis{\thesubsectiondis.\arabic{subsubsection}}


\hyphenation{op-tical net-works semi-conduc-tor}
\def\inputGnumericTable{}                                 %%

\lstset{
%language=C,
frame=single, 
breaklines=true,
columns=fullflexible
}
\begin{document}
\onecolumn

\newtheorem{theorem}{Theorem}[section]
\newtheorem{problem}{Problem}
\newtheorem{proposition}{Proposition}[section]
\newtheorem{lemma}{Lemma}[section]
\newtheorem{corollary}[theorem]{Corollary}
\newtheorem{example}{Example}[section]
\newtheorem{definition}[problem]{Definition}

\newcommand{\BEQA}{\begin{eqnarray}}
\newcommand{\EEQA}{\end{eqnarray}}
\newcommand{\define}{\stackrel{\triangle}{=}}
\bibliographystyle{IEEEtran}
\raggedbottom
\setlength{\parindent}{0pt}
\providecommand{\mbf}{\mathbf}
\providecommand{\pr}[1]{\ensuremath{\Pr\left(#1\right)}}
\providecommand{\qfunc}[1]{\ensuremath{Q\left(#1\right)}}
\providecommand{\sbrak}[1]{\ensuremath{{}\left[#1\right]}}
\providecommand{\lsbrak}[1]{\ensuremath{{}\left[#1\right.}}
\providecommand{\rsbrak}[1]{\ensuremath{{}\left.#1\right]}}
\providecommand{\brak}[1]{\ensuremath{\left(#1\right)}}
\providecommand{\lbrak}[1]{\ensuremath{\left(#1\right.}}
\providecommand{\rbrak}[1]{\ensuremath{\left.#1\right)}}
\providecommand{\cbrak}[1]{\ensuremath{\left\{#1\right\}}}
\providecommand{\lcbrak}[1]{\ensuremath{\left\{#1\right.}}
\providecommand{\rcbrak}[1]{\ensuremath{\left.#1\right\}}}
\theoremstyle{remark}
\newtheorem{rem}{Remark}
\newcommand{\sgn}{\mathop{\mathrm{sgn}}}
\providecommand{\abs}[1]{\left\vert#1\right\vert}
\providecommand{\res}[1]{\Res\displaylimits_{#1}} 
\providecommand{\norm}[1]{\left\lVert#1\right\rVert}
%\providecommand{\norm}[1]{\lVert#1\rVert}
\providecommand{\mtx}[1]{\mathbf{#1}}
\providecommand{\mean}[1]{E\left[ #1 \right]}
\providecommand{\fourier}{\overset{\mathcal{F}}{ \rightleftharpoons}}
%\providecommand{\hilbert}{\overset{\mathcal{H}}{ \rightleftharpoons}}
\providecommand{\system}{\overset{\mathcal{H}}{ \longleftrightarrow}}
	%\newcommand{\solution}[2]{\textbf{Solution:}{#1}}
\newcommand{\solution}{\noindent \textbf{Solution: }}
\newcommand{\cosec}{\,\text{cosec}\,}
\providecommand{\dec}[2]{\ensuremath{\overset{#1}{\underset{#2}{\gtrless}}}}
\newcommand{\myvec}[1]{\ensuremath{\begin{pmatrix}#1\end{pmatrix}}}
\newcommand{\mydet}[1]{\ensuremath{\begin{vmatrix}#1\end{vmatrix}}}
\numberwithin{equation}{subsection}
\makeatletter
\@addtoreset{figure}{problem}
\makeatother
\let\StandardTheFigure\thefigure
\let\vec\mathbf
\renewcommand{\thefigure}{\theproblem}
\def\putbox#1#2#3{\makebox[0in][l]{\makebox[#1][l]{}\raisebox{\baselineskip}[0in][0in]{\raisebox{#2}[0in][0in]{#3}}}}
     \def\rightbox#1{\makebox[0in][r]{#1}}
     \def\centbox#1{\makebox[0in]{#1}}
     \def\topbox#1{\raisebox{-\baselineskip}[0in][0in]{#1}}
     \def\midbox#1{\raisebox{-0.5\baselineskip}[0in][0in]{#1}}
\vspace{3cm}
\title{EE5609 Matrix Theory}
\author{Kranthi Kumar P}
\date{November 2020}
\maketitle
\bigskip
\renewcommand{\thefigure}{\theenumi}
\renewcommand{\thetable}{\theenumi}
%Download the python code for from 
%\begin{lstlisting}
%https://github.com/kranthiakssy/AI20RESCH14002_PhD_IITH/tree/master/EE5609_Matrix_Theory/Assignment-9
%\end{lstlisting}

Download the latex-file codes from 
\begin{lstlisting}
https://github.com/kranthiakssy/AI20RESCH14002_PhD_IITH/tree/master/EE5609_Matrix_Theory/Assignment-14
\end{lstlisting}
\section{\textbf{Problem}}
(ugcdec2014, 29) : \\
The determinant of n x n permutation matrix\\
$\myvec{ & & & & & &1\\ & & & & &1& \\ & & & &.& & \\ & & &.& & & \\ & &.& & & & \\ &1& & & & & \\1& & & & & &  }$
\begin{enumerate} %[label = (\alph*)]
\item $(-1)^n $
\item $(-1)^{\lfloor \frac{n}{2} \rfloor} $
\item $-1 $
\item $1$ 
\end{enumerate}
\section{\textbf{Solution}}
\renewcommand{\thetable}{1}
\begin{longtable}{|l|l|}
\hline
\multirow{3}{*}{Given} & \\
& n x n permutation matrix\\
& $\myvec{ & & & & & &1\\ & & & & &1& \\ & & & &.& & \\ & & &.& & & \\ & &.& & & & \\ &1& & & & & \\1& & & & & &  }$\\
&\\
\hline
\multirow{3}{*}{Solution} & \\
& The given n x n permutation matrix can be converted into \\
& identity matrix of n x n dimension by doing row exchange\\
& operations.\\
& \\
& from the row exchange property of determinants\\
& the determinant will by multiplied by -1 for every\\
& row exchange.\\
& \\
& the given n x n matrix requires $\lfloor \frac{n}{2} \rfloor$\\
& row exchanges to become identity matrix.\\
& \\
& Hence the determinant of given permutation matrix is\\
& $(-1)^{\lfloor \frac{n}{2} \rfloor}\begin{vmatrix}1& & & & & & \\ &1& & & & & \\ & &.& & & & \\ & & &.& & & \\ & & & &.& & \\ & & & & &1& \\ & & & & & &1  \end{vmatrix}$\\
& we know that the determinant of identity matrix, $det(\vec{I})=1$\\
& $\therefore$ the determinant of given n x n permutation matrix = $(-1)^{\lfloor \frac{n}{2} \rfloor}$\\
&\\
\hline
\multirow{3}{*}{Conclusion} & \\
& Option-2 is the right solution\\
&\\
\hline
\caption{Solution}
\label{table:1}
\end{longtable}
\section{\textbf{Example}}
\renewcommand{\thetable}{2}
\begin{longtable}{|l|l|}
\hline
\multirow{3}{*}{Example-1} & \\
& Let $\vec{A}$ is 5 x 5 permutation matrix, then\\
& $det(\vec{A})= \begin{vmatrix} & & & &1\\ & & &1& \\ & &1& & \\ &1& & & \\1& & & &    \end{vmatrix}$\\
& $ = \xleftrightarrow[]{R_1 \leftrightarrow R_5}
(-1)\begin{vmatrix}
1& & & & \\ & & &1& \\ & &1& & \\ &1& & & \\ & & & &1 
\end{vmatrix}$\\
& $ = \xleftrightarrow[]{R_2 \leftrightarrow R_4}
(-1)(-1)\begin{vmatrix}
1& & & & \\ &1& & & \\ & &1& & \\ & & &1& \\ & & & &1 
\end{vmatrix}$\\
& $ = 1$\\
& \\
& substituting n = 5 in the solution\\
& $(-1)^{\lfloor \frac{5}{2} \rfloor} = 1$\\
& \\
\hline
\multirow{3}{*}{Example-2} & \\
& Let $\vec{A}$ is 6 x 6 permutation matrix, then    \\
& $det(\vec{A})= \begin{vmatrix} & & & & &1\\ & & & &1& \\ & & &1& & \\ & &1& & & \\ &1& & & & \\1& & & & &    \end{vmatrix}$\\
& $ = \xleftrightarrow[]{R_1 \leftrightarrow R_6}
(-1)\begin{vmatrix}
1& & & & & \\ & & & &1& \\ & & &1& & \\ & &1& & & \\ &1& & & & \\ & & & & &1 
\end{vmatrix}$\\
& $ = \xleftrightarrow[]{R_2 \leftrightarrow R_5}
(-1)(-1)\begin{vmatrix}
1& & & & & \\ &1& & & & \\ & & &1& & \\ & &1& & & \\ & & & &1& \\ & & & & &1 
\end{vmatrix}$\\
& $ = \xleftrightarrow[]{R_3 \leftrightarrow R_4}
(-1)(-1)(-1)\begin{vmatrix}
1& & & & & \\ &1& & & & \\ & &1& & & \\ & & &1& & \\ & & & &1& \\ & & & & &1 
\end{vmatrix}$\\
& $ = -1$\\
& \\
& substituting n = 6 in the solution\\
& $(-1)^{\lfloor \frac{6}{2} \rfloor} = -1$\\
& \\
& Hence the proved that the solution is correct.\\
& \\
\hline
\caption{Example}
\label{table:2}
\end{longtable}
\end{document}