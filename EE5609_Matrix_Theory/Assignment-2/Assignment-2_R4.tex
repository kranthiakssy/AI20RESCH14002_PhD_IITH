\documentclass[journal,12pt,twocolumn]{IEEEtran}

\usepackage{setspace}
\usepackage{gensymb}

\singlespacing


\usepackage[cmex10]{amsmath}

\usepackage{amsthm}

\usepackage{mathrsfs}
\usepackage{txfonts}
\usepackage{stfloats}
\usepackage{bm}
\usepackage{cite}
\usepackage{cases}
\usepackage{subfig}

\usepackage{longtable}
\usepackage{multirow}

\usepackage{enumitem}
\usepackage{mathtools}
\usepackage{steinmetz}
\usepackage{tikz}
\usepackage{circuitikz}
\usepackage{verbatim}
\usepackage{tfrupee}
\usepackage[breaklinks=true]{hyperref}
\usepackage{graphicx}
\usepackage{tkz-euclide}

\usetikzlibrary{calc,math}
\usepackage{listings}
    \usepackage{color}                                            %%
    \usepackage{array}                                            %%
    \usepackage{longtable}                                        %%
    \usepackage{calc}                                             %%
    \usepackage{multirow}                                         %%
    \usepackage{hhline}                                           %%
    \usepackage{ifthen}                                           %%
    \usepackage{lscape}     
\usepackage{multicol}
\usepackage{chngcntr}

\DeclareMathOperator*{\Res}{Res}

\renewcommand\thesection{\arabic{section}}
\renewcommand\thesubsection{\thesection.\arabic{subsection}}
\renewcommand\thesubsubsection{\thesubsection.\arabic{subsubsection}}

\renewcommand\thesectiondis{\arabic{section}}
\renewcommand\thesubsectiondis{\thesectiondis.\arabic{subsection}}
\renewcommand\thesubsubsectiondis{\thesubsectiondis.\arabic{subsubsection}}


\hyphenation{op-tical net-works semi-conduc-tor}
\def\inputGnumericTable{}                                 %%

\lstset{
%language=C,
frame=single, 
breaklines=true,
columns=fullflexible
}
\begin{document}


\newtheorem{theorem}{Theorem}[section]
\newtheorem{problem}{Problem}
\newtheorem{proposition}{Proposition}[section]
\newtheorem{lemma}{Lemma}[section]
\newtheorem{corollary}[theorem]{Corollary}
\newtheorem{example}{Example}[section]
\newtheorem{definition}[problem]{Definition}

\newcommand{\BEQA}{\begin{eqnarray}}
\newcommand{\EEQA}{\end{eqnarray}}
\newcommand{\define}{\stackrel{\triangle}{=}}
\bibliographystyle{IEEEtran}
\providecommand{\mbf}{\mathbf}
\providecommand{\pr}[1]{\ensuremath{\Pr\left(#1\right)}}
\providecommand{\qfunc}[1]{\ensuremath{Q\left(#1\right)}}
\providecommand{\sbrak}[1]{\ensuremath{{}\left[#1\right]}}
\providecommand{\lsbrak}[1]{\ensuremath{{}\left[#1\right.}}
\providecommand{\rsbrak}[1]{\ensuremath{{}\left.#1\right]}}
\providecommand{\brak}[1]{\ensuremath{\left(#1\right)}}
\providecommand{\lbrak}[1]{\ensuremath{\left(#1\right.}}
\providecommand{\rbrak}[1]{\ensuremath{\left.#1\right)}}
\providecommand{\cbrak}[1]{\ensuremath{\left\{#1\right\}}}
\providecommand{\lcbrak}[1]{\ensuremath{\left\{#1\right.}}
\providecommand{\rcbrak}[1]{\ensuremath{\left.#1\right\}}}
\theoremstyle{remark}
\newtheorem{rem}{Remark}
\newcommand{\sgn}{\mathop{\mathrm{sgn}}}
\providecommand{\abs}[1]{\left\vert#1\right\vert}
\providecommand{\res}[1]{\Res\displaylimits_{#1}} 
\providecommand{\norm}[1]{\left\lVert#1\right\rVert}
%\providecommand{\norm}[1]{\lVert#1\rVert}
\providecommand{\mtx}[1]{\mathbf{#1}}
\providecommand{\mean}[1]{E\left[ #1 \right]}
\providecommand{\fourier}{\overset{\mathcal{F}}{ \rightleftharpoons}}
%\providecommand{\hilbert}{\overset{\mathcal{H}}{ \rightleftharpoons}}
\providecommand{\system}{\overset{\mathcal{H}}{ \longleftrightarrow}}
	%\newcommand{\solution}[2]{\textbf{Solution:}{#1}}
\newcommand{\solution}{\noindent \textbf{Solution: }}
\newcommand{\cosec}{\,\text{cosec}\,}
\providecommand{\dec}[2]{\ensuremath{\overset{#1}{\underset{#2}{\gtrless}}}}
\newcommand{\myvec}[1]{\ensuremath{\begin{pmatrix}#1\end{pmatrix}}}
\newcommand{\mydet}[1]{\ensuremath{\begin{vmatrix}#1\end{vmatrix}}}
\numberwithin{equation}{subsection}
\makeatletter
\@addtoreset{figure}{problem}
\makeatother
\let\StandardTheFigure\thefigure
\let\vec\mathbf
\renewcommand{\thefigure}{\theproblem}
\def\putbox#1#2#3{\makebox[0in][l]{\makebox[#1][l]{}\raisebox{\baselineskip}[0in][0in]{\raisebox{#2}[0in][0in]{#3}}}}
     \def\rightbox#1{\makebox[0in][r]{#1}}
     \def\centbox#1{\makebox[0in]{#1}}
     \def\topbox#1{\raisebox{-\baselineskip}[0in][0in]{#1}}
     \def\midbox#1{\raisebox{-0.5\baselineskip}[0in][0in]{#1}}
\vspace{3cm}
\title{EE5609 Matrix Theory}
\author{Kranthi Kumar P}
\date{September 2020}
\maketitle
\newpage
\bigskip
\renewcommand{\thefigure}{\theenumi}
\renewcommand{\thetable}{\theenumi}
Download the python code from 
\begin{lstlisting}
https://github.com/kranthiakssy/AI20RESCH14002_PhD_IITH/tree/master/EE5609_Matrix_Theory/Assignment-2
\end{lstlisting}
%
and latex-file codes from 
%
\begin{lstlisting}
https://github.com/kranthiakssy/AI20RESCH14002_PhD_IITH/tree/master/EE5609_Matrix_Theory/Assignment-2
\end{lstlisting}
\section*{Assignment-2}
\subsection*{Problem}
Lines and Planes (81) : In each of the following cases, determine the normal to the plane and the distance from the origin.
\begin{enumerate}[label = (\alph*)]
\item $\myvec{0&0&1}\vec{x} = 2 $
\item $\myvec{1&1&1}\vec{x} = 1 $
\item $\myvec{0&5&0}\vec{x} = -8 $
\item $\myvec{2&3&-1}\vec{x} = 5 $
\end{enumerate}
\subsection*{Solution}
If $ax+by+cz=k$ is a linear equation representing a plane, then the normal $\vec{n}$ to that plane is the coefficients of the linear equation
\begin{align}
\vec{n} = \myvec{a\\b\\c}
\end{align}
The equation of plane can be written as
\begin{align}
\vec{n}^T\vec{x} = c
\label{eq:a2}
\end{align}
The shortest distance between the plane \eqref{eq:a2} and origin is 
\begin{align}
\frac{\abs{c}}{\norm{\vec{n}}}
\label{eq:a3}
\end{align}
\begin{enumerate}[label = (\alph*)]

\item $\myvec{0&0&1}\vec{x} = 2 $
\begin{align}
\vec{n} = \myvec{0\\0\\1}\\
c = 2
\end{align}
As per \eqref{eq:a3}, shortest distance from origin = 
\begin{align}
\frac{\abs{2}}{\sqrt{0^2+0^2+1^2}} = 2
\end{align}

\item $\myvec{1&1&1}\vec{x} = 1 $
\begin{align}
\vec{n} = \myvec{1\\1\\1}\\
c = 1
\end{align}
As per \eqref{eq:a3}, shortest distance from origin = 
\begin{align}
\frac{\abs{1}}{\sqrt{1^2+1^2+1^1}} = \frac{1}{\sqrt{3}}
\end{align}

\item $\myvec{0&5&0}\vec{x} = -8 $
\begin{align}
\vec{n} = \myvec{0\\5\\0}\\
c = -8
\end{align}
As per \eqref{eq:a3}, shortest distance from origin = 
\begin{align}
\frac{\abs{-8}}{\sqrt{0^2+5^2+0^2}} = \frac{8}{5}
\end{align}

\item $\myvec{2&3&-1}\vec{x} = 5 $
\begin{align}
\vec{n} = \myvec{2\\3\\-1}\\
c = 5
\end{align}
As per \eqref{eq:a3}, shortest distance from origin = 
\begin{align}
\frac{\abs{5}}{\sqrt{2^2+3^2+(-1)^2}} = \frac{5}{\sqrt{14}}
\end{align}
%\begin{figure}
%\centering
%\includegraphics[width=\columnwidth]{assignment2.png}
%\caption{Plot showing the given two lines are parallel}
%\label{Fig 0}
%\end{figure}
\end{enumerate}
\end{document}