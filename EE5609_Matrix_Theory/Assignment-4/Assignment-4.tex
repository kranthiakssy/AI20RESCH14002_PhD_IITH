\documentclass[journal,12pt,twocolumn]{IEEEtran}

\usepackage{setspace}
\usepackage{gensymb}

\singlespacing


\usepackage[cmex10]{amsmath}

\usepackage{amsthm}

\usepackage{mathrsfs}
\usepackage{txfonts}
\usepackage{stfloats}
\usepackage{bm}
\usepackage{cite}
\usepackage{cases}
\usepackage{subfig}

\usepackage{longtable}
\usepackage{multirow}

\usepackage{enumitem}
\usepackage{mathtools}
\usepackage{steinmetz}
\usepackage{tikz}
\usepackage{circuitikz}
\usepackage{verbatim}
\usepackage{tfrupee}
\usepackage[breaklinks=true]{hyperref}
\usepackage{graphicx}
\usepackage{tkz-euclide}

\usetikzlibrary{calc,math}
\usepackage{listings}
    \usepackage{color}                                            %%
    \usepackage{array}                                            %%
    \usepackage{longtable}                                        %%
    \usepackage{calc}                                             %%
    \usepackage{multirow}                                         %%
    \usepackage{hhline}                                           %%
    \usepackage{ifthen}                                           %%
    \usepackage{lscape}     
\usepackage{multicol}
\usepackage{chngcntr}

\DeclareMathOperator*{\Res}{Res}

\renewcommand\thesection{\arabic{section}}
\renewcommand\thesubsection{\thesection.\arabic{subsection}}
\renewcommand\thesubsubsection{\thesubsection.\arabic{subsubsection}}

\renewcommand\thesectiondis{\arabic{section}}
\renewcommand\thesubsectiondis{\thesectiondis.\arabic{subsection}}
\renewcommand\thesubsubsectiondis{\thesubsectiondis.\arabic{subsubsection}}


\hyphenation{op-tical net-works semi-conduc-tor}
\def\inputGnumericTable{}                                 %%

\lstset{
%language=C,
frame=single, 
breaklines=true,
columns=fullflexible
}
\begin{document}


\newtheorem{theorem}{Theorem}[section]
\newtheorem{problem}{Problem}
\newtheorem{proposition}{Proposition}[section]
\newtheorem{lemma}{Lemma}[section]
\newtheorem{corollary}[theorem]{Corollary}
\newtheorem{example}{Example}[section]
\newtheorem{definition}[problem]{Definition}

\newcommand{\BEQA}{\begin{eqnarray}}
\newcommand{\EEQA}{\end{eqnarray}}
\newcommand{\define}{\stackrel{\triangle}{=}}
\bibliographystyle{IEEEtran}
\providecommand{\mbf}{\mathbf}
\providecommand{\pr}[1]{\ensuremath{\Pr\left(#1\right)}}
\providecommand{\qfunc}[1]{\ensuremath{Q\left(#1\right)}}
\providecommand{\sbrak}[1]{\ensuremath{{}\left[#1\right]}}
\providecommand{\lsbrak}[1]{\ensuremath{{}\left[#1\right.}}
\providecommand{\rsbrak}[1]{\ensuremath{{}\left.#1\right]}}
\providecommand{\brak}[1]{\ensuremath{\left(#1\right)}}
\providecommand{\lbrak}[1]{\ensuremath{\left(#1\right.}}
\providecommand{\rbrak}[1]{\ensuremath{\left.#1\right)}}
\providecommand{\cbrak}[1]{\ensuremath{\left\{#1\right\}}}
\providecommand{\lcbrak}[1]{\ensuremath{\left\{#1\right.}}
\providecommand{\rcbrak}[1]{\ensuremath{\left.#1\right\}}}
\theoremstyle{remark}
\newtheorem{rem}{Remark}
\newcommand{\sgn}{\mathop{\mathrm{sgn}}}
\providecommand{\abs}[1]{\left\vert#1\right\vert}
\providecommand{\res}[1]{\Res\displaylimits_{#1}} 
\providecommand{\norm}[1]{\left\lVert#1\right\rVert}
%\providecommand{\norm}[1]{\lVert#1\rVert}
\providecommand{\mtx}[1]{\mathbf{#1}}
\providecommand{\mean}[1]{E\left[ #1 \right]}
\providecommand{\fourier}{\overset{\mathcal{F}}{ \rightleftharpoons}}
%\providecommand{\hilbert}{\overset{\mathcal{H}}{ \rightleftharpoons}}
\providecommand{\system}{\overset{\mathcal{H}}{ \longleftrightarrow}}
	%\newcommand{\solution}[2]{\textbf{Solution:}{#1}}
\newcommand{\solution}{\noindent \textbf{Solution: }}
\newcommand{\cosec}{\,\text{cosec}\,}
\providecommand{\dec}[2]{\ensuremath{\overset{#1}{\underset{#2}{\gtrless}}}}
\newcommand{\myvec}[1]{\ensuremath{\begin{pmatrix}#1\end{pmatrix}}}
\newcommand{\mydet}[1]{\ensuremath{\begin{vmatrix}#1\end{vmatrix}}}
\numberwithin{equation}{subsection}
\makeatletter
\@addtoreset{figure}{problem}
\makeatother
\let\StandardTheFigure\thefigure
\let\vec\mathbf
\renewcommand{\thefigure}{\theproblem}
\def\putbox#1#2#3{\makebox[0in][l]{\makebox[#1][l]{}\raisebox{\baselineskip}[0in][0in]{\raisebox{#2}[0in][0in]{#3}}}}
     \def\rightbox#1{\makebox[0in][r]{#1}}
     \def\centbox#1{\makebox[0in]{#1}}
     \def\topbox#1{\raisebox{-\baselineskip}[0in][0in]{#1}}
     \def\midbox#1{\raisebox{-0.5\baselineskip}[0in][0in]{#1}}
\vspace{3cm}
\title{EE5609 Matrix Theory}
\author{Kranthi Kumar P}
\date{September 2020}
\maketitle
\newpage
\bigskip
\renewcommand{\thefigure}{\theenumi}
\renewcommand{\thetable}{\theenumi}
Download the python code from 
\begin{lstlisting}
https://github.com/kranthiakssy/AI20RESCH14002_PhD_IITH/tree/master/EE5609_Matrix_Theory/Assignment-4
\end{lstlisting}
%
and latex-file codes from 
%
\begin{lstlisting}
https://github.com/kranthiakssy/AI20RESCH14002_PhD_IITH/tree/master/EE5609_Matrix_Theory/Assignment-4
\end{lstlisting}
\section*{Assignment-4}
\subsection*{Problem:}
Determinants (79):\\
Using properties of determinants, prove that:\\[6pt]
$\begin{vmatrix}
x&x^2&1+px^3\\y&y^2&1+py^3\\z&z^2&1+pz^3
\end{vmatrix}
= (1+pxyz)(x-y)(y-z)(z-x)$

\subsection*{Solution:}
\begin{align}
LHS=\begin{vmatrix}
x&x^2&1+px^3\\y&y^2&1+py^3\\z&z^2&1+pz^3
\end{vmatrix}
\end{align}
By expanding using sum property
\begin{align}
=\begin{vmatrix}
x&x^2&1\\y&y^2&1\\z&z^2&1
\end{vmatrix}
+\begin{vmatrix}
x&x^2&px^3\\y&y^2&py^3\\z&z^2&pz^3
\end{vmatrix}
\end{align}
By using switching of rows(or columns) property
\begin{align}
=(-1)\begin{vmatrix}
1&x^2&x\\1&y^2&y\\1&z^2&z
\end{vmatrix}
+\begin{vmatrix}
x&x^2&px^3\\y&y^2&py^3\\z&z^2&pz^3
\end{vmatrix}\\
=(-1)^2\begin{vmatrix}
1&x&x^2\\1&y&y^2\\1&z&z^2
\end{vmatrix}
+\begin{vmatrix}
x&x^2&px^3\\y&y^2&py^3\\z&z^2&pz^3
\end{vmatrix}
\end{align}
By using scalar multiplication property
\begin{align}
=\begin{vmatrix}
1&x&x^2\\1&y&y^2\\1&z&z^2
\end{vmatrix}
+(pxyz)\begin{vmatrix}
1&x&x^2\\1&y&y^2\\1&z&z^2
\end{vmatrix}\\
=(1+pxyz)\begin{vmatrix}
1&x&x^2\\1&y&y^2\\1&z&z^2
\end{vmatrix}
\end{align}
By applying row reduction
\begin{align}
=(1+pxyz)\begin{vmatrix}
1&x&x^2\\1&y&y^2\\1&z&z^2
\end{vmatrix}\\
 \xleftrightarrow[]{R_2 \leftarrow R_2-R_3}
(1+pxyz)\begin{vmatrix}
1&x&x^2\\0&y-z&y^2-z^2\\1&z&z^2
\end{vmatrix}\\
 \xleftrightarrow[]{R_3 \leftarrow R_3-R_1}
(1+pxyz)\begin{vmatrix}
1&x&x^2\\0&y-z&y^2-z^2\\0&z-x&z^2-x^2
\end{vmatrix}
\end{align}
By using scalar multiplication property
\begin{align}
=(1+pxyz)(y-z)(z-x)\begin{vmatrix}
1&x&x^2\\0&1&y+z\\0&1&z+x
\end{vmatrix}
\end{align}
By applying the determinant formula
\begin{align}
=(1+pxyz)(y-z)(z-x)(z+x-y-z)\\
=(1+pxyz)(x-y)(y-z)(z-x)\\
= RHS
\end{align}
Hence Proved.
%\begin{figure}
%\centering
%\includegraphics[width=\columnwidth]{assignment2.png}
%\caption{Plot showing the given two lines are parallel}
%\label{Fig 0}
%\end{figure}
\end{document}