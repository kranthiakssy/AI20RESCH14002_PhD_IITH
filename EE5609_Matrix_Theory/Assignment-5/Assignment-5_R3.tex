\documentclass[journal,12pt,twocolumn]{IEEEtran}

\usepackage{setspace}
\usepackage{gensymb}

\singlespacing


\usepackage[cmex10]{amsmath}

\usepackage{amsthm}

\usepackage{mathrsfs}
\usepackage{txfonts}
\usepackage{stfloats}
\usepackage{bm}
\usepackage{cite}
\usepackage{cases}
\usepackage{subfig}

\usepackage{longtable}
\usepackage{multirow}

\usepackage{enumitem}
\usepackage{mathtools}
\usepackage{steinmetz}
\usepackage{tikz}
\usepackage{circuitikz}
\usepackage{verbatim}
\usepackage{tfrupee}
\usepackage[breaklinks=true]{hyperref}
\usepackage{graphicx}
\usepackage{tkz-euclide}

\usetikzlibrary{calc,math}
\usepackage{listings}
    \usepackage{color}                                            %%
    \usepackage{array}                                            %%
    \usepackage{longtable}                                        %%
    \usepackage{calc}                                             %%
    \usepackage{multirow}                                         %%
    \usepackage{hhline}                                           %%
    \usepackage{ifthen}                                           %%
    \usepackage{lscape}     
\usepackage{multicol}
\usepackage{chngcntr}

\DeclareMathOperator*{\Res}{Res}

\renewcommand\thesection{\arabic{section}}
\renewcommand\thesubsection{\thesection.\arabic{subsection}}
\renewcommand\thesubsubsection{\thesubsection.\arabic{subsubsection}}

\renewcommand\thesectiondis{\arabic{section}}
\renewcommand\thesubsectiondis{\thesectiondis.\arabic{subsection}}
\renewcommand\thesubsubsectiondis{\thesubsectiondis.\arabic{subsubsection}}


\hyphenation{op-tical net-works semi-conduc-tor}
\def\inputGnumericTable{}                                 %%

\lstset{
%language=C,
frame=single, 
breaklines=true,
columns=fullflexible
}
\begin{document}


\newtheorem{theorem}{Theorem}[section]
\newtheorem{problem}{Problem}
\newtheorem{proposition}{Proposition}[section]
\newtheorem{lemma}{Lemma}[section]
\newtheorem{corollary}[theorem]{Corollary}
\newtheorem{example}{Example}[section]
\newtheorem{definition}[problem]{Definition}

\newcommand{\BEQA}{\begin{eqnarray}}
\newcommand{\EEQA}{\end{eqnarray}}
\newcommand{\define}{\stackrel{\triangle}{=}}
\bibliographystyle{IEEEtran}
\providecommand{\mbf}{\mathbf}
\providecommand{\pr}[1]{\ensuremath{\Pr\left(#1\right)}}
\providecommand{\qfunc}[1]{\ensuremath{Q\left(#1\right)}}
\providecommand{\sbrak}[1]{\ensuremath{{}\left[#1\right]}}
\providecommand{\lsbrak}[1]{\ensuremath{{}\left[#1\right.}}
\providecommand{\rsbrak}[1]{\ensuremath{{}\left.#1\right]}}
\providecommand{\brak}[1]{\ensuremath{\left(#1\right)}}
\providecommand{\lbrak}[1]{\ensuremath{\left(#1\right.}}
\providecommand{\rbrak}[1]{\ensuremath{\left.#1\right)}}
\providecommand{\cbrak}[1]{\ensuremath{\left\{#1\right\}}}
\providecommand{\lcbrak}[1]{\ensuremath{\left\{#1\right.}}
\providecommand{\rcbrak}[1]{\ensuremath{\left.#1\right\}}}
\theoremstyle{remark}
\newtheorem{rem}{Remark}
\newcommand{\sgn}{\mathop{\mathrm{sgn}}}
\providecommand{\abs}[1]{\left\vert#1\right\vert}
\providecommand{\res}[1]{\Res\displaylimits_{#1}} 
\providecommand{\norm}[1]{\left\lVert#1\right\rVert}
%\providecommand{\norm}[1]{\lVert#1\rVert}
\providecommand{\mtx}[1]{\mathbf{#1}}
\providecommand{\mean}[1]{E\left[ #1 \right]}
\providecommand{\fourier}{\overset{\mathcal{F}}{ \rightleftharpoons}}
%\providecommand{\hilbert}{\overset{\mathcal{H}}{ \rightleftharpoons}}
\providecommand{\system}{\overset{\mathcal{H}}{ \longleftrightarrow}}
	%\newcommand{\solution}[2]{\textbf{Solution:}{#1}}
\newcommand{\solution}{\noindent \textbf{Solution: }}
\newcommand{\cosec}{\,\text{cosec}\,}
\providecommand{\dec}[2]{\ensuremath{\overset{#1}{\underset{#2}{\gtrless}}}}
\newcommand{\myvec}[1]{\ensuremath{\begin{pmatrix}#1\end{pmatrix}}}
\newcommand{\mydet}[1]{\ensuremath{\begin{vmatrix}#1\end{vmatrix}}}
\numberwithin{equation}{subsection}
\makeatletter
\@addtoreset{figure}{problem}
\makeatother
\let\StandardTheFigure\thefigure
\let\vec\mathbf
\renewcommand{\thefigure}{\theproblem}
\def\putbox#1#2#3{\makebox[0in][l]{\makebox[#1][l]{}\raisebox{\baselineskip}[0in][0in]{\raisebox{#2}[0in][0in]{#3}}}}
     \def\rightbox#1{\makebox[0in][r]{#1}}
     \def\centbox#1{\makebox[0in]{#1}}
     \def\topbox#1{\raisebox{-\baselineskip}[0in][0in]{#1}}
     \def\midbox#1{\raisebox{-0.5\baselineskip}[0in][0in]{#1}}
\vspace{3cm}
\title{EE5609 Matrix Theory}
\author{Kranthi Kumar P}
\date{September 2020}
\maketitle
\newpage
\bigskip
\renewcommand{\thefigure}{\theenumi}
\renewcommand{\thetable}{\theenumi}
%Download the python code from 
%\begin{lstlisting}
%https://github.com/kranthiakssy/AI20RESCH14002_PhD_IITH/tree/master/%EE5609_Matrix_Theory/Assignment-5
%\end{lstlisting}
%
Download the latex-file codes from 
%
\begin{lstlisting}
https://github.com/kranthiakssy/AI20RESCH14002_PhD_IITH/tree/master/EE5609_Matrix_Theory/Assignment-5
\end{lstlisting}
\section*{Assignment-5\\geolin}
\subsection*{Problem:}
Triangle Exercises (1.19):\\
D is a point on side BC of $\triangle ABC $ such that AD = AC. Show that AB $>$ AD
\subsection*{Solution:}
\begin{figure}[!ht]
\centering
\resizebox{\columnwidth}{!}{\documentclass{standalone}

\usepackage{tikz,pgf}
\usepackage{tkz-euclide} % loads  TikZ and tkz-base
%\usetkzobj{all}
\usetikzlibrary{calc,math}
\begin{document}

%\begin{tikzpicture}

%\draw (0,0) coordinate (B) node[anchor=north]{$B$}
%  -- (4,4) coordinate (A) node[anchor=west]{$A$}
%  -- (6,0) coordinate (C) node[anchor=north]{$C$}
%  -- cycle;
%\draw(2,0) node[anchor=north]{$D$}
%  -- (4,4)
%  -- cycle;
%\end{tikzpicture}
\begin{tikzpicture}
[scale=1,>=stealth,point/.style={draw,circle,fill = black,inner sep=0.5pt},]

%Triangle sides
\def\a{4}
\def\b{6}
\def\c{2}



%Labeling points
\node (A) at (\a,\a)[point,label=above right:$A$] {};
\node (B) at (0, 0)[point,label=below left:$B$] {};
\node (C) at (\b, 0)[point,label=below right:$C$] {};
%\node (M) at (\a*0.5,\b*0.5)[point,label=above:$M$] {};
\node (D) at (\c, 0)[point,label=below:$D$] {};
\coordinate [label={below:$1$}] (1) at (1,0);
\coordinate [label={below:$k$}] (k) at (4,0);

%Drawing triangle ABC
\draw (A) -- node[left] {$\textrm{}$} (B) -- node[below] {$\textrm{}$} (C) -- node[above,xshift=1.5mm] {$\textrm{}$} (A);

%Joining AD
\draw (A)--node[above,xshift=3mm] {$\textrm{}$}(D);
%Joining BD
%\draw (B)--(D);

%Drawing and marking angles
%\tkzMarkAngle[fill=orange!40,size=0.5cm,mark=](A,M,C)
%\tkzMarkAngle[fill=orange!40,size=0.5cm,mark=](B,M,D)
%\tkzMarkAngle[fill=green!40,size=0.5cm,mark=](A,B,D)
%\tkzMarkRightAngle[fill=blue!20,size=.2](A,C,B)
%\tkzMarkRightAngle[fill=blue!20,size=.2](D,B,C)
%\tkzLabelAngle[pos=0.65](A,M,C){$\theta$}
%\tkzLabelAngle[pos=0.65](B,M,D){$\theta$}
%\tkzLabelAngle[pos=0.65](A,D,B){$\alpha$}


\end{tikzpicture}
\end{document}}
\caption{Triangle generated using LaTeX-Tikz}
\label{fig:tri}	
\end{figure}
The above Fig. \ref{fig:tri} shows that, point D placed on side BC of $\triangle ABC$ such that 
\begin{align}
\norm{\vec{D-A}} = \norm{\vec{C-A}}
\label{eq:a0}
\end{align}
Let Point D bisecting the side BC at 1:k ratio and\\
Direction vectors of AB,AD \& AC are\\
$\vec{B-A}, \vec{D-A} \& \vec{C-A}$ respectively.\\
By applying the section formula for bisecting the line internally 
\begin{align}
\vec{(D-A)} = \frac{k.\vec{(B-A)}+ 1.\vec{(C-A)}}{1+k}
\end{align}
\begin{align}
\implies \norm{\vec{D-A}}^2=\frac{k^2 \norm{\vec{B-A}}^2 + \norm{\vec{C-A}}^2}{(1+k)^2}
\end{align}
substituting \eqref{eq:a0}
\begin{align}
\norm{\vec{D-A}}^2=\frac{k^2 \norm{\vec{B-A}}^2}{(1+k)^2}+\frac{\norm{\vec{D-A}}^2}{(1+k)^2}\\
\implies \norm{\vec{D-A}}^2 \left(1-\frac{1}{(1+k)^2}\right)=\frac{k^2 \norm{\vec{B-A}}^2}{(1+k)^2}\\
\implies \norm{\vec{D-A}}^2 \left(1+\frac{2}{k}\right) = \norm{\vec{B-A}}^2\\
\therefore AB > AD 
\end{align}
for $k > 0$\\
Hence Proved.
%\begin{figure}
%\centering
%\includegraphics[width=\columnwidth]{assignment2.png}
%\caption{Plot showing the given two lines are parallel}
%\label{Fig 0}
%\end{figure}
\end{document}