\documentclass[journal,12pt,twocolumn]{IEEEtran}

\usepackage{setspace}
\usepackage{gensymb}

\singlespacing


\usepackage[cmex10]{amsmath}

\usepackage{amsthm}

\usepackage{mathrsfs}
\usepackage{txfonts}
\usepackage{stfloats}
\usepackage{bm}
\usepackage{cite}
\usepackage{cases}
\usepackage{subfig}

\usepackage{longtable}
\usepackage{multirow}

\usepackage{enumitem}
\usepackage{mathtools}
\usepackage{steinmetz}
\usepackage{tikz}
\usepackage{circuitikz}
\usepackage{verbatim}
\usepackage{tfrupee}
\usepackage[breaklinks=true]{hyperref}
\usepackage{graphicx}
\usepackage{tkz-euclide}

\usetikzlibrary{calc,math}
\usepackage{listings}
    \usepackage{color}                                            %%
    \usepackage{array}                                            %%
    \usepackage{longtable}                                        %%
    \usepackage{calc}                                             %%
    \usepackage{multirow}                                         %%
    \usepackage{hhline}                                           %%
    \usepackage{ifthen}                                           %%
    \usepackage{lscape}     
\usepackage{multicol}
\usepackage{chngcntr}

\DeclareMathOperator*{\Res}{Res}

\renewcommand\thesection{\arabic{section}}
\renewcommand\thesubsection{\thesection.\arabic{subsection}}
\renewcommand\thesubsubsection{\thesubsection.\arabic{subsubsection}}

\renewcommand\thesectiondis{\arabic{section}}
\renewcommand\thesubsectiondis{\thesectiondis.\arabic{subsection}}
\renewcommand\thesubsubsectiondis{\thesubsectiondis.\arabic{subsubsection}}


\hyphenation{op-tical net-works semi-conduc-tor}
\def\inputGnumericTable{}                                 %%

\lstset{
%language=C,
frame=single, 
breaklines=true,
columns=fullflexible
}
\begin{document}


\newtheorem{theorem}{Theorem}[section]
\newtheorem{problem}{Problem}
\newtheorem{proposition}{Proposition}[section]
\newtheorem{lemma}{Lemma}[section]
\newtheorem{corollary}[theorem]{Corollary}
\newtheorem{example}{Example}[section]
\newtheorem{definition}[problem]{Definition}

\newcommand{\BEQA}{\begin{eqnarray}}
\newcommand{\EEQA}{\end{eqnarray}}
\newcommand{\define}{\stackrel{\triangle}{=}}
\bibliographystyle{IEEEtran}
\providecommand{\mbf}{\mathbf}
\providecommand{\pr}[1]{\ensuremath{\Pr\left(#1\right)}}
\providecommand{\qfunc}[1]{\ensuremath{Q\left(#1\right)}}
\providecommand{\sbrak}[1]{\ensuremath{{}\left[#1\right]}}
\providecommand{\lsbrak}[1]{\ensuremath{{}\left[#1\right.}}
\providecommand{\rsbrak}[1]{\ensuremath{{}\left.#1\right]}}
\providecommand{\brak}[1]{\ensuremath{\left(#1\right)}}
\providecommand{\lbrak}[1]{\ensuremath{\left(#1\right.}}
\providecommand{\rbrak}[1]{\ensuremath{\left.#1\right)}}
\providecommand{\cbrak}[1]{\ensuremath{\left\{#1\right\}}}
\providecommand{\lcbrak}[1]{\ensuremath{\left\{#1\right.}}
\providecommand{\rcbrak}[1]{\ensuremath{\left.#1\right\}}}
\theoremstyle{remark}
\newtheorem{rem}{Remark}
\newcommand{\sgn}{\mathop{\mathrm{sgn}}}
\providecommand{\abs}[1]{\left\vert#1\right\vert}
\providecommand{\res}[1]{\Res\displaylimits_{#1}} 
\providecommand{\norm}[1]{\left\lVert#1\right\rVert}
%\providecommand{\norm}[1]{\lVert#1\rVert}
\providecommand{\mtx}[1]{\mathbf{#1}}
\providecommand{\mean}[1]{E\left[ #1 \right]}
\providecommand{\fourier}{\overset{\mathcal{F}}{ \rightleftharpoons}}
%\providecommand{\hilbert}{\overset{\mathcal{H}}{ \rightleftharpoons}}
\providecommand{\system}{\overset{\mathcal{H}}{ \longleftrightarrow}}
	%\newcommand{\solution}[2]{\textbf{Solution:}{#1}}
\newcommand{\solution}{\noindent \textbf{Solution: }}
\newcommand{\cosec}{\,\text{cosec}\,}
\providecommand{\dec}[2]{\ensuremath{\overset{#1}{\underset{#2}{\gtrless}}}}
\newcommand{\myvec}[1]{\ensuremath{\begin{pmatrix}#1\end{pmatrix}}}
\newcommand{\mydet}[1]{\ensuremath{\begin{vmatrix}#1\end{vmatrix}}}
\numberwithin{equation}{subsection}
\makeatletter
\@addtoreset{figure}{problem}
\makeatother
\let\StandardTheFigure\thefigure
\let\vec\mathbf
\renewcommand{\thefigure}{\theproblem}
\def\putbox#1#2#3{\makebox[0in][l]{\makebox[#1][l]{}\raisebox{\baselineskip}[0in][0in]{\raisebox{#2}[0in][0in]{#3}}}}
     \def\rightbox#1{\makebox[0in][r]{#1}}
     \def\centbox#1{\makebox[0in]{#1}}
     \def\topbox#1{\raisebox{-\baselineskip}[0in][0in]{#1}}
     \def\midbox#1{\raisebox{-0.5\baselineskip}[0in][0in]{#1}}
\vspace{3cm}
\title{EE5609 Matrix Theory}
\author{Kranthi Kumar P}
\date{September 2020}
\maketitle
\newpage
\bigskip
\renewcommand{\thefigure}{\theenumi}
\renewcommand{\thetable}{\theenumi}
Download the python code for from 
\begin{lstlisting}
https://github.com/kranthiakssy/AI20RESCH14002_PhD_IITH/tree/master/EE5609_Matrix_Theory/Assignment-9
\end{lstlisting}

Download the latex-file codes from 
%
\begin{lstlisting}
https://github.com/kranthiakssy/AI20RESCH14002_PhD_IITH/tree/master/EE5609_Matrix_Theory/Assignment-9
\end{lstlisting}
\section*{Assignment-9}
\subsection*{Problem:}
SVD:\\
Find the foot of the perpendicular to the plane
\begin{align}
2x+3y-2z+4 = 0 \label{eq:equation}
\end{align}
from the point $\myvec{3\\-2\\0}$ using SVD.
\subsection*{Solution:}
The given plane equation is
\begin{align}
\myvec{2&3&-2}\vec{x} = 0\label{eq:genform}\\
\end{align}
The equation of plane is
\begin{align}
\vec{n}^T\vec{x} = c
\end{align}
Hence the normal vector $\vec{n}$ is,
\begin{align}
\vec{n} = \myvec{2\\3\\-2}
\end{align}
Let, the normal vectors $\vec{m_1}$ and $\vec{m_2}$ to the normal vector $\vec{n}$ be, 
\begin{align}
\vec{m} = \myvec{a\\b\\c}\\
\mbox{then, } \vec{m}^T\vec{n} = 0\\
\implies \myvec{a&b&c}\myvec{2\\3\\-2} = 0
\end{align}  
Let, a=1 and b=0 we get,
\begin{align}
\vec{m_1} = \myvec{1\\0\\1}
\end{align}
Let, a=0 and b=1,
\begin{align}
\vec{m_2} = \myvec{0\\1\\\frac{3}{2}}
\end{align}
Now solving the equation,
\begin{align}
\vec{Mx} = \vec{b}
\label{eq:SVD_M}
\end{align}
Where,
\begin{align}
\vec{M} = \myvec{1&0\\0&1\\1&\frac{3}{2}} , & \vec{b} = \myvec{3\\-2\\0}\label{eq:b}
\end{align}
To solve \eqref{eq:SVD_M} we perform singular value decomposition on M given by, 
\begin{align}
\vec{M} = \vec{US}\vec{V}^T
\label{eq:SVD}
\end{align}
substituting the value of $\vec{M}$ from equation \eqref{eq:SVD} to \eqref{eq:SVD_M},
\begin{align}
\implies \vec{US}\vec{V}^T\vec{x} = \vec{b}\\
\implies \vec{x} = \vec{VS}_+\vec{U}^T\vec{b} \label{eq:x}
\end{align}
where, $\vec{S}_+$ is Moore-Pen-rose Pseudo-Inverse of $\vec{S}$.\\
Columns of $\vec{U}$ are eigenvectors of $\vec{MM}^T$, columns of $\vec{V}$ are eigenvectors of $\vec{M}^T\vec{M}$ and $\vec{S}$ is diagonal matrix of singular value of eigenvalues of $\vec{M}^T\vec{M}$.\\
First calculating the eigenvectors corresponding to $\vec{M}^T\vec{M}$.
\begin{align}
\vec{M}^T\vec{M} =  \myvec{1&0&1\\0&1&\frac{3}{2}} \myvec{1&0\\0&1\\1&\frac{3}{2}} = \myvec{2&\frac{3}{2}\\\frac{3}{2}&\frac{13}{4}}
\end{align}
Eigenvalues corresponding to $\vec{M}^T\vec{M}$  is,
\begin{align}
\mydet{\vec{M}^T\vec{M}-\lambda\vec{I}} = 0\\
\implies \myvec{2-\lambda&\frac{3}{2}\\\frac{3}{2}&\frac{13}{4}-\lambda}\\
\implies (\lambda-\frac{17}{4})(\lambda-1) = 0\\
\therefore \lambda_1 = \frac{17}{4}, \lambda_2 = 1, \label{eq:lambda2}
\end{align} 
Hence the eigenvectors corresponding to $\lambda_1$ and $\lambda_2$ respectively is,
\begin{align}
\vec{v_1} =\myvec{\frac{2}{3}\\1}, \vec{v_2} =\myvec{\frac{-3}{2}\\1}
\end{align}
Normalizing the eigenvectors we get,
\begin{align}
\vec{v_1} = \frac{1}{\sqrt{13}}\myvec{2\\3}\\
\vec{v_2} = \frac{1}{\sqrt{13}}\myvec{-3\\2}\\
\implies \vec{V} = \frac{1}{\sqrt{13}}\myvec{2&-3\\3&2}\label{eq:V}
\end{align}
Now calculating the eigenvectors corresponding to $\vec{MM}^T$
\begin{align}
\vec{MM}^T = \myvec{1&0\\0&1\\1&\frac{3}{2}}\myvec{1&0&1\\0&1&\frac{3}{2}} \\\implies \myvec{1&0&1\\0&1&\frac{3}{2}\\1&\frac{3}{2}&\frac{13}{4}}
\end{align}
Eigenvalues corresponding to $\vec{M}\vec{M}^T$  is,
\begin{align}
\mydet{\vec{M}\vec{M}^T-\lambda\vec{I}} = 0\\
\implies \myvec{1-\lambda&0&1\\0&1-\lambda&\frac{3}{2}\\1&\frac{3}{2}&\frac{13}{4}-\lambda}\\
\implies \lambda(\lambda-1)(\lambda-\frac{17}{4}) = 0\\
\therefore \lambda_3 = \frac{17}{4}, \lambda_4 = 1, \lambda_5 = 0\label{eq:lambda5}
\end{align} 
Hence the eigenvectors corresponding to $\lambda_3$, $\lambda_4$ and  $\lambda_5$ respectively is,
\begin{align}
\vec{v_3} =\myvec{\frac{4}{13}\\\frac{6}{13}\\1},
\vec{v_4} =\myvec{\frac{-3}{2}\\1\\0},
\vec{v_5} = \myvec{-1\\\frac{-3}{2}\\1}
\end{align}
Normalizing the eigenvectors we get,
\begin{align}
\vec{v_3} = \myvec{\frac{4}{\sqrt{221}}\\\frac{6}{\sqrt{221}}\\\frac{13}{\sqrt{221}}}\\
\vec{v_4} = \myvec{\frac{-3}{\sqrt{13}}\\\frac{2}{\sqrt{13}}\\0}\\
\vec{v_5} = \myvec{\frac{-2}{\sqrt{17}}\\\frac{-3}{\sqrt{17}}\\\frac{2}{\sqrt{17}}}\\
\implies \vec{U} = \myvec{\frac{4}{\sqrt{221}} &\frac{-3}{\sqrt{13}} & \frac{-2}{\sqrt{17}}\\\frac{6}{\sqrt{221}} &\frac{2}{\sqrt{13}} &\frac{-3}{\sqrt{17}} \\\frac{13}{\sqrt{221}}&0& \frac{2}{\sqrt{17}}} \label{eq:U}
\end{align} 
Now $\vec{S}$ corresponding to eigenvalues $\lambda_3$, $\lambda_4$ and  $\lambda_5$ is as follows,
\begin{align}
\vec{S} = \myvec{\sqrt{\frac{17}{4}}&0\\0&1\\0&0}
\end{align}
Now, Moore-Penrose Pseudo inverse of $\vec{S}$ is given by,
\begin{align}
\vec{S}_+ = \myvec{\frac{2}{\sqrt{17}}&0&0\\0&1&0}\label{eq:S+}
\end{align}
Hence we get singular value decomposition of $\vec{M}$ as,
\begin{align}
\vec{M} = \frac{1}{\sqrt{13}}\myvec{\frac{4}{\sqrt{221}} &\frac{-3}{\sqrt{13}} & \frac{-2}{\sqrt{17}}\\\frac{6}{\sqrt{221}} &\frac{2}{\sqrt{13}} &\frac{-3}{\sqrt{17}} \\\frac{13}{\sqrt{221}}&0& \frac{2}{\sqrt{17}}}\myvec{\sqrt{\frac{17}{4}}&0\\0&1\\0&0}\myvec{2&-3\\3&2}^T
\end{align}
Now substituting the values of \eqref{eq:V}, \eqref{eq:S+}, \eqref{eq:U} and \eqref{eq:b} in \eqref{eq:x},
\begin{align}
\vec{U}^T\vec{b} = \myvec{\frac{4}{\sqrt{221}} &\frac{-3}{\sqrt{13}} & \frac{-2}{\sqrt{17}}\\\frac{6}{\sqrt{221}} &\frac{2}{\sqrt{13}} &\frac{-3}{\sqrt{17}} \\\frac{13}{\sqrt{221}}&0& \frac{2}{\sqrt{17}}}^T\myvec{3\\-2\\0}\\
\implies \vec{U}^T\vec{b} = \myvec{0\\-\sqrt{13}\\0}\label{eq:Utb}
\end{align}
\begin{align}
\vec{S}_+\vec{U}^T\vec{b} = \myvec{\frac{2}{\sqrt{17}}&0&0\\0&1&0}\myvec{0\\-\sqrt{13}\\0}\\
\implies \vec{S}_+\vec{U}^T\vec{b} = \myvec{0\\-\sqrt{13}}\label{eq:SUtb}
\end{align}
\begin{align}
\vec{V}\vec{S}_+\vec{U}^T\vec{b} = \frac{1}{\sqrt{13}}\myvec{2&-3\\3&2}\myvec{0\\-\sqrt{13}}\\
\implies \vec{V}\vec{S}_+\vec{U}^T\vec{b} = \myvec{3\\-2}
\end{align}
$\therefore$ from equation \eqref{eq:x},
\begin{align}
\vec{x} = \myvec{3 \\ -2 } \label{eq:x_svd}
\end{align}
Verifying the solution using,
\begin{align}
\vec{M}^T\vec{Mx} = \vec{M}^T\vec{b}
\end{align}
\begin{align}
\implies \myvec{1&0&1\\0&1&\frac{3}{2}} \myvec{1&0\\0&1\\1&\frac{3}{2}} \vec{x} = \myvec{1&0&1\\0&1&\frac{3}{2}}\myvec{3\\-2\\0}\\
\implies \myvec{2&\frac{3}{2}\\\frac{3}{2}&\frac{13}{4}}\vec{x} = \myvec{3\\-2}
\end{align}
Solving the augmented matrix we get,
\begin{align}
\myvec{2&\frac{3}{2}&3\\\frac{3}{2}&\frac{13}{4}&-2}\xleftrightarrow[]{R_1\leftarrow\frac{R_1}{2}}\myvec{1 &\frac{3}{4}&\frac{3}{2}\\\frac{3}{2}&\frac{13}{4}&-2}\\
\xleftrightarrow[]{R_2\leftarrow R_2-\frac{3}{2}R_1}\myvec{1&\frac{3}{4}&\frac{3}{2}\\0&\frac{17}{8}&-\frac{17}{4}}\\
\xleftrightarrow[]{R_2\leftarrow \frac{8}{17}R_2}\myvec{1&\frac{3}{4}&\frac{3}{2}\\0&1&-2}\\
\xleftrightarrow[]{R_1\leftarrow R_1-\frac{3}{4}R_2}\myvec{1&0&3\\0&1&-2}\\
\implies \vec{x} = \myvec{3\\-2 }\label{eq:x_rref}
\end{align}
Hence from equations \eqref{eq:x_svd} and \eqref{eq:x_rref} we conclude that the solution is verified.
\end{document}